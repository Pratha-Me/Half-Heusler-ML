\documentclass{article}

% set font encoding for PDFLaTeX, XeLaTeX, or LuaTeX
\usepackage{ifxetex,ifluatex}
\if\ifxetex T\else\ifluatex T\else F\fi\fi T%
  \usepackage{fontspec}
\else
  \usepackage[T1]{fontenc}
  \usepackage[utf8]{inputenc}
  \usepackage{lmodern}
\fi

\usepackage{hyperref}
\usepackage{cite}

\usepackage[top=0.75in, bottom=1in, left=1in, right=1in]{geometry}

\title{Machine learning based prediction of magnetic properties for half‑Heusler compounds using atomic information}
\author{Pramosh Shrestha}

% Enable SageTeX to run SageMath code right inside this LaTeX file.
% http://doc.sagemath.org/html/en/tutorial/sagetex.html
% \usepackage{sagetex}

% Enable PythonTeX to run Python – https://ctan.org/pkg/pythontex
% \usepackage{pythontex}

\begin{document}
\maketitle

\begin{abstract}
Due to the myriad of applicability the Heusler alloys receive tremendous experimental and theoretical interest. We are particularly interested in its use for spintronics and as a material. For the same, the calculation of total molecular magnetic moment and elasticity of the compound are the crucial physical parameters.Traditionally, these calculations are performed via costly DFT. In this paper, we aim to demonstrate high reliability of machine learning model for the sought after parameters. By the end of this paper, we achieved our aim by simply deploying the atomic information of constituent elements like atomic radii and atomic mass. It is easily assumed the ML model trained on the half-Heusler compounds can easily incorporate the full-Heusler compounds. With that in mind the study is carried out on the half-Heusler compounds to predict magnetic moment and elasticity. Our results may serve as a motivation to harness the robust ML models into calculation of desired parameters. Which may lead to a low‑cost and short time development of new functional devices and materials.
\end{abstract}

\section{Introduction}
A general understanding of how microscopic structures and characteristics gives forth the material properties is a necessary prerequisite for designing next iteration of desired materials. By and large for this effort, the typical, computational simulations like DFT and molecular dynamics are vital tools. So far the Heusler compounds are treated in the equal footing\cite{srivastava2020investigation, khandy2017dft}. However, in recent times the emerging Machine Learning techniques have grabbed the attention for such objectives. Sprung from the domain of statistics, the ML techniques have seeped into multiple disciplines including the material science. The cubic ${X_2YZ}$ half-Heusler compounds are candidate for spintronics technology \cite{de1983new, feng2014first, zhang2017two, ma2017computational, dehghan2019d0}. Interestingly, they possess high spin polarisation. The net electron spin magnetic moment in the molecule is an instrumental parameter for spintronics. We are inquisitive to see how well will a ML model closes down to the values calculated from DFT. Especially, after several reports of successful implementation of ML models to improve the thermal properties of materials \cite{wan2019materials, ouyang2020accuracy, ouyang2021machine, wang2021prediction, miyazaki2021machine}.

The magnetism in a material originates from electron spin and the electron exchange interactions\cite{dietl2009exchange, klitzing:qhe, commins:qhe}. Indeed due to the double exchange, the compound ${Cu_2MnAl}$ becomes ferromagnetic, although non of its constituent elements is ferromagnetic by itself\cite{heusler1903magnetisch}. The presence of two different magnetic sublattices in the crystal structure, the ${XYZ}$ Heusler compounds displays multiple magnetic phenomena, and in fact, today ferromagnetism, ferrimagnetism, and half-metallic ferromagnetism are well studied. Similarly, the half-Heusler, materials exhibit one magnetic sublattice since only the atoms on the octahedral sites can carry a magnetic moment \cite{graf2013magnetic}. Understanding of these electronic properties have led to great breakthroughs. 

Another important probe in new material designs and improvements is Elastic modulus. Several physical properties, such as hardness and melting point, are related to the elastic constants \cite{fine1984elastic, gilman2009chemistry}. For the Heusler compounds, elasticity has been studied to determine its stability\cite{wu2017critical}. In this study, it will be the next parameter to predict via ML model.

As of today, Machine Learning algorithms like Decision trees and Deep neural networks has been applied to classify the Topological insulators\cite{claussen2020detection, andrejevic2020machine} and topological phases\cite{ming2019quantum}. Moreover, the high accuracy in lattice parameters prediction are demonstrated in multiple studies\cite{chonghe2003prediction, jiang2006prediction, zhang2020machine}. Machine learning techniques may soon be poised to compete, or at least corroborate, the results from traditional Computational frameworks. After all, it's robust and requires few and simple training features like atomic configurations and molecular parameters to churn out desired parameters. Incidentally, It's not the everyday choice. However, without any hassle the input features for any ML model can be easily be increased for better output. The algorithms and accuracy are getting better with every new study. In this paper, we tried to achieve similar feat on Machine learning based studies of materials. The necessary training data is obtained from the Python library Pymatgen\cite{ong2013python}.

\bibliographystyle{abbrv}
\bibliography{ref.bib}

\end{document}

